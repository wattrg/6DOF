\documentclass[11pt, twoside]{article}

\usepackage[margin = 2cm]{geometry} % sets margins
\setlength{\headheight}{14pt}
\usepackage{fancyhdr} % for headers
\usepackage{amsmath,amsthm} % math packages
\usepackage{graphicx,float}% for importing and placing figures
\graphicspath{./img/}
\usepackage{titlesec} %for customizing headings
\usepackage{genmpage} % for using minipages
\usepackage{booktabs, multirow, multicol, array}
\usepackage{multicol}
\usepackage{subcaption}
\usepackage[separate-uncertainty = true,multi-part-units=single]{siunitx}

\usepackage{parskip}
\usepackage{listings}
\lstset{
	breaklines = true
	language = Python
}



%\setlength\parindent{0pt} % removes indent at beginning of paragraph

\fancyhf{} %resests page numbering
\pagestyle{fancy} %sets page style
\lhead{6 DOF solver} % left header
\rhead{Robert Watt - 44311513} % right header
\cfoot{\thepage}

\usepackage{hyperref}%Completely useless for hard copy, but pretty schmick nonetheless 
\hypersetup{
	colorlinks=True,
	linkcolor=black,
	filecolor=black,      
	urlcolor=black,
	citecolor=black,
}

\numberwithin{equation}{section}
%\usepackage[backend=biber, labeldate=year]{biblatex}
%\DeclareLanguageMapping{british}{british-apa}%\addbibresource{./bib.tex}
%\setlength\bibitemsep{0.5\baselineskip}


\titleformat{\section}{\bfseries}{\thesection.0}{0.5em}{} %Sets section header format
\titleformat{\subsection}{\bfseries\small}{\thesubsection}{0.5em}{}

\newcommand{\Appendix}{%set appendix header format (start the appendix 		section with \Appendix)
	\appendix %begins the appendix section
	\titleformat{\section}{\large\bfseries}{Appendix \thesection}{0.5em}{} %sets appendix header style
	\titleformat{\subsection}{\bfseries}{Appendix \thesubsection}{0.5em}{} % sets sub appendix header style
}

\renewcommand{\vec}{\mathbf}

\newcolumntype{P}[1]{>{\centering\arraybackslash}p{#1}}

\begin{document}

\section{Equations of Motion}
The state of the object in question solved in body fixed coordinates. The velocity and angular velocity of the object is \(\vec{v}_b = \left[u \, v \, w\right]^T\), and \( \vec{\omega}_b = \left[ p \,  q \, r \right]^T \) respectively. The inertial (flat earth) reference frame is denoted as \(\vec{v}_I\), and the position of the centre of mass of the boomerang in the inertial frame is \(\vec{x}_I\) The moment of inertia is given as:

\begin{align}
	I = 
	\begin{bmatrix}
		I_{xx} & I_{xy} & I_{xz} \\
		I_{yx} & I_{yy} & I_{yz} \\
		I_{zx} & I_{zy} & I_{zz}
	\end{bmatrix}
\end{align}

Newton's second law applied to the body fixed coordinates gives

\begin{align}
	\vec{F} = m \frac{\mathrm{d} \vec{v}_b }{\mathrm{d}t} + m\vec{\omega}_b \times \vec{v}_b & & \vec{M} =\frac{\mathrm{d}\vec{H}_b}{\mathrm{d}t} + \vec{\omega}_b \times \vec{H}_b
\end{align}

Where \(\vec{H}_b\), the anglular momentum is given by

\begin{align}
	\vec{H}_b = I\vec{\omega}_b
\end{align}

Expanding and rearranging the transnational equations of motion,

\begin{align}
	\frac{\mathrm{d}}{\mathrm{d}t}
	\begin{bmatrix}
		u \\ v \\ w
	\end{bmatrix}
	=
	\frac{1}{m}
	\begin{bmatrix}
		F_x \\ F_y \\ F_z
	\end{bmatrix}
	-
	\begin{bmatrix}
		qw - rv \\ ru -pw \\ pv - qu
	\end{bmatrix}
\end{align}

Expanding and rearranging the rotational equations of motion:

\begin{align}
	I \frac{\mathrm{d}\vec{\omega}}{\mathrm{d}t} &= \vec{M} - \vec{\omega} \times (I\vec{\omega})\\
	\frac{\mathrm{d}}{\mathrm{d}t}
	\begin{bmatrix}
		p \\ q \\ r
	\end{bmatrix} &= 
	I^{-1} \left( \begin{bmatrix}
		M_x \\ M_y \\ M_z
	\end{bmatrix}	 
	- \left[\begin{array}{c} q \left(I_{zx} p+I_{zy} q + I_{zz} r\right)-r \left(I_{yx} p + I_{yy} q + I_{yz} r \right)\\
	 r \left(I_{xx} p + I_{xy} q + I_{xz} r \right) - p\left(I_{zx} p + I_{zy} q + I_{zz} r \right)\\
	  p \left(I_{yx} p + I_{yy} q + I_{yz} r \right) - q\left(I_{xx} p + I_{xy}  q + I_{xz} r \right) \end{array}\right]
 \right)
\end{align}

Note that \(I^{-1}\) is constant in time, and so can be calculated ahead of time, reducing computation.

The transformation matrix between the body fixed frame, and the body fixed frame at the initial state is given by \(T\)

\begin{align}
	T = 
	\begin{bmatrix}
		\cos\theta \cos\psi & \cos\theta \sin\psi & -\sin\theta \\
		-\cos\phi \sin\psi + \sin\phi \sin\theta \cos\psi & \cos\phi \cos\psi + \sin\phi \sin\theta \cos\psi & \sin\phi \cos\theta \\
		\sin\phi \sin\psi + \cos\phi \sin\theta \cos\psi & \sin\phi \cos\psi + \cos\phi \sin\theta \cos\psi & \cos\phi \cos\theta
	\end{bmatrix}
\end{align}

Thus a vector in the inertial frame can be converted into the body fixed frame by the successive transformation \(T_0 T\), where \(T_0\) is \(T\) evaluated at the initial state. The transformation can be applied in reverse by inverting \(T\). Since \(T\) is a unitary matrix, its inverse can be evaluated by taking its transpose.

The primary variables determined during the integration are \(\vec{v}_b\), \(\vec{\omega}_b\) and \(\vec{\varepsilon}\). From these, the velocity and position in the inertial frame are also calculated.


\section{Aerodynamics}

The direction in which drag acts (in the body fixed frame) is opposite to the direction of motion (in the body fixed frame), \(-\vec{v}_b\).

To determine the direction of lift, we need 3 constraints, one for each component of lift. These constraints are:
\begin{itemize}
	\item Lift and drag act at right angles to each other, so \(\hat{\vec{L}} \cdot \hat{\vec{D}} = 0\)
	\item The lift, drag, and the \(z\) axis of the body fitted reference frame are co-planar (this ensures that as the body rolls the lift force also rotates), so \(\hat{\vec{z}} \cdot \left(\hat{\vec{L}} \times \hat{\vec{D}}\right)=0\)
	\item A constraint on the length. Since ensuring unit length introduces non-linear terms to the set of equations, so we will introduce a dummy constraint, then scale the vector afterwards to ensure unit length, so \(L_1 + L_2 + L_3 = 1\).
\end{itemize}

Solving for the 3 components of the lift ``unit" vector, with \(\hat{\vec{z}} = \begin{bmatrix}
0 & 0 & 1
\end{bmatrix}\) gives

\begin{align}
	\hat{\vec{L}} = 
	\begin{bmatrix}
		-\frac{(D_1D_3)}{(D_1^2 - D_3D_1 + D_2^2 - D_3D_2)}\\
      -\frac{(D_2D_3)}{(D_1^2 - D_3D_1 + D_2^2 - D_3D_2)}\\
 \frac{(D_1^2 + D_2^2)}{(D_1^2 - D_3D_1 + D_2^2 - D_3D_2)}
	\end{bmatrix}
\end{align}

\end{document}